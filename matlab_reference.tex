%
% Some information based on MATLAB for Engineers by Holly More
% Some information based on  Giiordano Fusco's MATLAB Reference Sheet (http://userpage.chemie.fu-berlin.de/~naundorf/PCF/MATLAB_Reference_Sheet-1.pdf)
% Based on Winston's LateX Cheat Sheet http://stdout.org/~winston/latex/
%

\documentclass[10pt,landscape]{article}
\usepackage{multicol}
\usepackage{calc}
\usepackage{ifthen}
\usepackage[landscape]{geometry}


% This sets page margins to .5 inch if using letter paper, and to 1cm
% if using A4 paper. (This probably isn't strictly necessary.)
% If using another size paper, use default 1cm margins.
\ifthenelse{\lengthtest { \paperwidth = 11in}}
	{ \geometry{top=.5in,left=.5in,right=.5in,bottom=.5in} }
	{\ifthenelse{ \lengthtest{ \paperwidth = 297mm}}
		{\geometry{top=1cm,left=1cm,right=1cm,bottom=1cm} }
		{\geometry{top=1cm,left=1cm,right=1cm,bottom=1cm} }
	}

% Turn off header and footer
\pagestyle{empty}
 

% Redefine section commands to use less space
\makeatletter
\renewcommand{\section}{\@startsection{section}{1}{0mm}%
                                {-1ex plus -.5ex minus -.2ex}%
                                {0.5ex plus .2ex}%x
                                {\normalfont\large\bfseries}}
\renewcommand{\subsection}{\@startsection{subsection}{2}{0mm}%
                                {-1explus -.5ex minus -.2ex}%
                                {0.5ex plus .2ex}%
                                {\normalfont\normalsize\bfseries}}
\renewcommand{\subsubsection}{\@startsection{subsubsection}{3}{0mm}%
                                {-1ex plus -.5ex minus -.2ex}%
                                {1ex plus .2ex}%
                                {\normalfont\small\bfseries}}
\makeatother

% Define BibTeX command
\def\BibTeX{{\rm B\kern-.05em{\sc i\kern-.025em b}\kern-.08em
    T\kern-.1667em\lower.7ex\hbox{E}\kern-.125emX}}

% Don't print section numbers
\setcounter{secnumdepth}{0}


\setlength{\parindent}{0pt}
\setlength{\parskip}{0pt plus 0.5ex}


% -----------------------------------------------------------------------

\begin{document}

\raggedright
\footnotesize
\begin{multicols}{3}


% multicol parameters
% These lengths are set only within the two main columns
%\setlength{\columnseprule}{0.25pt}
\setlength{\premulticols}{1pt}
\setlength{\postmulticols}{1pt}
\setlength{\multicolsep}{1pt}
\setlength{\columnsep}{2pt}

\begin{center}
     \Large{\textbf{MATLAB Reference Sheet}\\ Anthony Kosednar - MAE 215}
\end{center}

\section{IEEE-754}
\begin{tabular}{@{}ll@{}}
sign (s), mantissa (m), base (b), and exponent (e)\\
\textbf{Formula:} \begin{math}(-1)^{s} * m * (b)^{e}\end{math} \\
\end{tabular}

\section{MATLAB Tid Bits}
\begin{tabular}{@{}ll@{}}
MATLAB stands for Matrix Laboratory\\
MATLAB was first written in FORTAN\\
MATLAB was later rewritten in C\\
1 and l look similar but are different\\
1 is not a prime number in MATLAB\\
\end{tabular}

\section{Basic Types}
\begin{tabular}{@{}ll@{}}
A scalar is a 1\verb!x!1 array.\\
A row vector of length n is a 1\verb!x!n array.\\
A column vector of length m is an m\verb!x!1 array.\\
A matrix of dimensions m rows and n columns is an m\verb!x!n array.\\
\end{tabular}

\section{Naming Conventions}
\begin{tabular}{@{}ll@{}}
Case sensitive variable names\\
Must start with a letter (A-Z,a-z).\\
Up to 31 letters, digits, and underscores\\
No special characters besides underscores.\\
Can not be a reserved keyword.\\
\end{tabular}

\section{Reserved Keywords}
\begin{verbatim}
    break
    case
    catch
    classdef
    continue
    else
    elseif
    end
    for
    function
    global
    if
    otherwise
    parfor
    persistent
    return
    spmd
    switch
    try
    while
\end{verbatim}

\section{Special Characters}
\begin{tabular}{@{}ll@{}}
\verb![]!    & forms matrices \\
\verb!()!  & used in statements to group operations \\
\verb!.! & decimal point \\
\verb!,!  & separates subscripts or matrix elements\\
\verb!;! & separates rows in a matrix definition or suppresses output \\
\verb!:! & indicates all rows or all columns \\
\verb!=! & assignment operator (not equality) \\
\verb!%! & indicates a comment \\
\verb!%%! & cell divider \\
\verb!+! &addition \\
\verb!-! & ubtraction \\
\verb!*! & multiplication \\
\verb!.*! & array multiplication \\
\verb!/! & division \\
\verb!./! & array division \\
\verb!^! & exponential \\
\verb!.^! & array exponential \\
\end{tabular}

\section{Special Variables/Constants}
\begin{tabular}{@{}ll@{}}
\verb!ans!    & default variable for last calculation \\
\verb!eps!  & 1\\
\verb!flops! & count of �oating point operations \\
\verb!Inf!  & infinity\\
\verb!NaNa! & NaN\\
\verb!pi! & the math pi (3.1415�) \\
\verb!i,j! & \begin{math}\sqrt[]{-1}\end{math}\\
\verb!realmax! & largest real number MATLAB can represent\\
\verb!realmin! & smallest real number MATLAB can represent\\
\verb!intmax! & returns largest possible integer used in MATLAB\\
\verb!intmin! & returns smallest possible integer used in MATLAB\\
\verb!clock! & returns the time\\
\verb!date! & returns the date\\
\end{tabular}

\section{Basic Commands and Functions}
\begin{tabular}{@{}ll@{}}
\verb!Clc!    & clears command window \\
\verb!clear!  & clears workspace\\
\verb!Diary! & creates a copy of all commands and most results \\
\verb!exit!  & terminates MATLAB\\
\verb!help! & invokes help utility\\
\verb!helpwin! & invokes windowed help utility\\
\verb!length! & length of a vector or maximum dimension of an array\\
\verb!size! & display dimensions of a particular array\\
\verb!quite! & terminates MATLAB \\
\verb!save! & saves variables in a file \\
\verb!who! & lists  variables in memory\\
\verb!whos! & lists  variable names, sizes, and types in memory\\
\end{tabular}

\section{Relational and Logical Operators}
\begin{tabular}{@{}ll@{}}
\verb!<!    & less than \\
\verb!<==!    & less than or equal to \\
\verb!>!    & greater than \\
\verb!>==!    & greater than or equal to \\
\verb!==!    &  equal to \\
\verb!~=!    &  not equal to \\
\verb!&!    &  and \\
\verb!!!    &  or \\
\verb!~!    &  not \\
\end{tabular}

\section{Conditional Statements}
\begin{verbatim}
if expression
   statements
elseif expression
   statements
else
   statements
end

switch switch_expression
   case case_expression
      statements
   case case_expression
      statements
   otherwise
      statements
end
\end{verbatim}

\section{Loops}
\begin{verbatim}
for k = vectorOrColumnList
   statements
end

while logicalExpression
   statements
end
\end{verbatim}

\section{User Defined Functions}
\begin{tabular}{@{}ll@{}}
\verb!normal - function [out1,out2] = func_name(inp1,inp2)..end!    & \\
\verb!anonymous - var_name = @(inp1,inpt2) (inp1*inp2)!    & \\
\end{tabular}


\section{Vector/Matrix Stuff}
\begin{tabular}{@{}ll@{}}
\verb!linspace(a,b,N)!    & spaced intervals btwn a \verb!&! b comprised of N pts\\
\verb!zeros(m,n)!    & an m by n array of zeroes\\
\verb!zeros(n)!    & an n by n array of zeroes\\
\verb!ones(m,n)!    & an m by n array of ones\\
\verb!ones(n)!    & an n by n array of ones\\
\verb!eye(m,n)!    &  an m by n array with ones on the diagonal\\
\verb!eye(n)!    &  an n by n identity matrix\\
\verb!ones(n)!    &  an n by n array of ones\\
\verb!rand(m,n)!    &  an m by n array of random numbers\\
\verb!rand(n)!    &  an n by n array of random numbers\\
\verb!i:k:l!    &  list generation: 1stV alue : Stride : LastV alue\\
\verb!v(1)!    &  1st element of vector v\\
\verb!v(end)!    &  last element of vector v\\
\verb!v(1:2:9)!    &  1st, 3rd, 5th, 7th, 9th elements of vector v\\
\verb!v(2:3:9)!    &  2nd, 5th, 8th elements of vector v\\
\verb!A(2,3)!    &  2�nd row, 3�rd column of matrix A\\
\verb!A(:,3)!    &  all elements in column 3\\
\verb!A(1,:)!    &  all elements in row 1\\
\verb!A(1:2:end,:)!    &  all odd rows of matrix A\\
\verb!A(1:2,2:4)!    &  sub-matrix of rows 1 and 2, columns 2 through 4\\
\verb!A(1,end)!    &  last element in 1�st row\\
\end{tabular}

\rule{0.3\linewidth}{0.25pt}
\scriptsize

Copyright \copyright\ 2013 Anthony Kosednar, Ver: 1.0 \newline
Based on the works of Holly Moore, and Giordano Fusco\newline
Format based on Winston Chang's LaTeX Cheat Sheet\newline

\end{multicols}
\end{document}